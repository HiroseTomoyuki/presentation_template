\documentclass[unicode,12pt]{beamer}

\usepackage{luatexja}
\renewcommand{\kanjifamilydefault}{\gtdefault}

\usetheme{metropolis}

\usepackage{graphicx}
\usepackage{listings}
\lstset{
  basicstyle=\small\ttfamily,
  identifierstyle=\small,
  ndkeywordstyle=\small,
  tabsize=4,
  frame={shadowbox},
  frameround={ffff},
  breaklines=true,
  columns=[l]{fullflexible},
  numbers=left,
  numbersep=5pt,
  numberstyle=\scriptsize,
  stepnumber=1,
  lineskip=-0.5ex
}
\renewcommand{\lstlistingname}{ソースコード} % キャプション名の変更

\usepackage{hyperref}
\usepackage{here}

\title{スライドサンプル}
\author{OSSS B4 広瀬智之}
\date{\today}

\begin{document}

\begin{frame}[plain]
  \titlepage
\end{frame}

\begin{frame}[plain]{目次}
  \tableofcontents
\end{frame}

\section{セクション1}
\begin{frame}[plain]{Hoge}
  セクション1の内容
  セクション1の内容
  セクション1の内容
  セクション1の内容
  セクション1の内容
  セクション1の内容
\end{frame}

\section{セクション2}
\begin{frame}[fragile]{コード}
\begin{lstlisting}[language=c,caption=サンプルコード,label=code:sample]
#include <stdio.h>

int main(void)
{
    printf("hello, %s\n", "world!");
    return 0;
}
\end{lstlisting}
\end{frame}

\end{document}
